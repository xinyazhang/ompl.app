\chapter{Advanced Topics in OMPL}

This chapter explains some of the more advanced features of OMPL.  This section
is not meant to explain all of the expert level concepts, but to give the user
an idea about some commonly used optional features of the library and how they
can be included in the code.  API documentation is updated regularly at
{\tt ompl.kavrakilab.org}, and includes the following concepts, among others.

\section {State Space Construction}
OMPL provides implementations for several common states spaces, including
${\mathbb R}^n$ for euclidean spaces and SO(2) and SO(3) for the space of
rotations in 2D and 3D respectively.  Many systems operate in spaces
that are combinations of ${\mathbb R}^n$ and orientations.  For example, the
classic ``piano mover's'' problem operates in SE(3), which is a combination
of ${\mathbb R}^3$ and SO(3).  OMPL contains an implementation of the SE(2) and
SE(3) state spaces for such instances.

In complex systems, like highly articulated robot arms, it is infeasible to
enumerate all possible state spaces.  However, the state spaces in many of these
systems can be decomposed into smaller spaces (like the SE(3) example).  To
accommodate this, OMPL allows for the construction of such spaces through the
CompoundStateSpace class, which allows state spaces to be formed from a
combination of their subcomponents.  The existing state spaces all provide
operator overloads for the `+' and `$-$', allowing readable and intuitive code to
be written when constructing such spaces.

\section {Planner Customization}
In the motion planning literature, there exist many variants of the classical
planning algorithms that change only one component of the planning algorithm
in order to achieve benefits in certain instances.  Many of the planners that
come bundled with OMPL are highly customizable, and in some cases it isn't
necessary to derive a new planner from the existing code base to recreate one of
these variations or test a new idea or algorithm.

An easy example is the nearest neighbor search performed in nearly all of the
planners to find the closest existing state to a newly generated random sample.
There are many algorithms to perform this search---both exact and
approximate---and evaluating the best approach for one particular system can become time
consuming.  The {\tt PRM} and {\tt RRT} planners (among others) both provide a
hook to replace the default nearest neighbor search with a user supplied
object that derives from the abstract NearestNeighbor class.  This allows users
to develop compact and portable components for differing nearest neighbor
schemes and test them without creating unnecessary planner objects that are
simply one-off from one another.  Similar possibilities exist not only with
nearest neighbor searching, but also in other planning components, like state
sampling and connection strategies.

\section {Python Bindings}
Users of OMPL are not bound to C++.  Most of the API is exposed through a set
of Python bindings from which users can create Python scripts to solve motion
planning queries.  These bindings are compiled separately from the OMPL core
library, and require Py++ and Boost.Python for correct operation.  Ensure that
these dependencies exist before installing OMPL to create the bindings. The
Python bindings are still considered an experimental feature and all API
documentation is for the C++ API (although we have tried to keep the python API
as similar as possible). The easiest way to get started with the python bindings
is by looking at a few of the demo programs.

The bindings are separated by ompl namespaces into python modules contained
within one parent {\tt ompl} module.  For example, to use a geometric motion
planner from OMPL, the {\tt base} and {\tt geometric} modules must be imported:
\\
{\tt from ompl import base\\
from ompl import geometric}

There are some subtle and some not-so-subtle differences between the C++ and
Python APIs due to the differences between the two programming languages.  A
user checking the OMPL C++ API must note the following discrepancies when
writing a program in Python:

\begin {itemize}
\item An STL vector of int's is of type vectorInt in Python (analogously for
other types).
\item The C++ class State has been renamed AbstractState, while the C++ class
ScopedState<> is called State in Python.
\item The C++ class ScopedState<RealVectorStateSpace> is called RealVectorState.
The ScopedState's for the other pre-defined state spaces have been renamed
analogously.
\item The C++ class RealVectorStateSpace::StateType has been renamed to
RealVectorStateInternal (analogously for the other state space types), to
emphasize that a user should really be using RealVectorState.
\item Use the ``()'' operator on a ScopedState in Python to retrieve a reference
to a C++ State.
\item The print method (for classes that have one) is mapped to the special
python method \_\_str\_\_, so a C++ call like {\tt foo.print(std::cout)} becomes
{\tt print foo} in python. Similarly, a C++ call like
{\tt foo.printSettings(std::cout)} becomes {\tt print foo.settings()} in python.
\item The signature of the state validity checker is changed. In python the
state validity checker has the following form:
\begin {verbatim}
def isStateValid(spaceInformation, state):
    return spaceInformation.satiesfiesBounds(state)
\end {verbatim}

\end {itemize}

